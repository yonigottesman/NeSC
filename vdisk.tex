

%from micro2015template:
\documentclass{sig-alternate}

% Set letter paper size:
\setlength{\paperheight}{11in}
\setlength{\paperwidth}{8.5in}
\usepackage[
  pass,% keep layout unchanged 
  % showframe,% show the layout
]{geometry}

\newcommand{\ignore}[1]{}
\usepackage{fancyhdr}
\usepackage[normalem]{ulem}
\usepackage[hyphens]{url}
\usepackage{hyperref}

% from Yoav's template
\usepackage{color}
\usepackage{xcolor}
\usepackage{graphicx}

%%%%%%%%%%%---SETME-----%%%%%%%%%%%%%
\newcommand{\microsubmissionnumber}{XXX}
%%%%%%%%%%%%%%%%%%%%%%%%%%%%%%%%%%%%

\fancypagestyle{firstpage}{
  \fancyhf{}
\setlength{\headheight}{50pt}
\renewcommand{\headrulewidth}{0pt}
  \fancyhead[C]{\normalsize{MICRO 2016 Submission
      \textbf{\#\microsubmissionnumber} -- Confidential Draft -- Do NOT Distribute!!}} 
  \pagenumbering{arabic}
}  

\usepackage{tabularx}
\usepackage{tabulary}
% more compact + self-contained (with embedded fonts)
\usepackage[T1]{fontenc}
\usepackage[ansinew]{inputenc}
\usepackage{pslatex}
%\usepackage[normalem]{ulem}
%\usepackage{titling}            % control the maketitle
%\setlength{\droptitle}{-3.3em}  % This is your set screw
\usepackage[kerning,spacing]{microtype}
%\microtypecontext{spacing=nonfrench}
\usepackage[caption=false]{subfig}
%\captionsetup{position=top} % put subcations above subfigures
\captionsetup[subfloat]{margin=3pt}

% util
\newcommand{\mypar}[1]{\textbf{#1}\ \ }

% yoav macros
\newcommand{\hide}[1]{}
\setlength{\marginparwidth}{2cm}
%\newcommand{\comment}[1]{\marginpar{\footnotesize #1}}

\newcommand{\comment}[1]{}
%\newcommand{\comment}[1]{\textcolor{red}{[#1]}}

\newcommand{\replace}[2]{
  \textcolor{red}{#1}

  \textcolor{yellow}{#2}
}


\renewcommand{\tilde}[0]{$\sim$}
\newcommand{\us}[0]{$\mu s$}

\newcommand{\speedup}[1]{#1$\times$}
\newcommand{\percent}[1]{#1\% }

%###############################################################################
\title{NeSC: Self Virtualizing Nested Storage Controller}
%###############################################################################

\begin{document}
\maketitle
\thispagestyle{firstpage}
\pagestyle{plain}


\author{
  {\rm
  Yonatan Gottesman$^{\dagger}$
  Yoav Etsion$^{\dagger}$}\\[0mm]
  $^\dagger$Technion---Israel Institute of Technology
}

%% %###############################################################################
%% \begin{document}
%% \date{}
%% \maketitle


%###############################################################################
% abstract
%###############################################################################
\begin{abstract}
The emergence of high-speed, multi GB/s storage devices has shifted the performance bottleneck of storage virtualization to the software layers of the hypervisor. The hypervisor overheads can be avoided by allowing the virtual machine (VM) to directly access the storage device (a method known as direct device assignment), but this method voids all protection guarantees provided by  filesystem permissions, since the device has no notion of client isolation.
Recently, following the introduction of 10Gbs and higher networking interfaces, the PCI-SIG was extended to include the SR-IOV specification for self-virtualizing devices, which allows a single physical device to present multiple virtual interfaces on the PCIe interconnect. Using SR-IOV, a hypervisor can directly assign a virtual PCIe device interface to each of its VMs. However, as networking interfaces simply multiplex packets sent from/to different clients, the specification does not dictate the semantics of a virtual storage device and how to maintain data isolation in a self virtualizing device.



  In this paper we present the self-virtualizing, nested storage controller (NeSC) architecture, which includes a filesystem-agnostic protection mechanism that enables the physical device to export files as virtual PCIe storage devices. The protection mechanism maps file offsets to physical blocks and thereby offloads the hypervisor's storage layer functionality to hardware.
  Using NeSC, a hypervisor can securely expose its files as virtual PCIe devices and directly assign them to VMs.
  We have prototyped a 1GB/s NeSC controller using a Virtex-7 FPGA development board connected to the PCIe interconnect. 
  Our evaluation of NeSC on a real system shows that NeSC virtual devices enable VMs to access to their data with near-native performance (in terms of both throughput and latency).
\end{abstract}

\replace{
  this is the original paragraph.
}{
  this is the new paragraph.
}


%###############################################################################
%%%%%%%%%%%%%%%%%%%%%%%%%%%%%%%%%%%%%%%%%%%%%%%%%%%%%%%%%%%%%%%%%%%%%%%%
\section{Introduction}
\label{sec:intro}

%%%%%%%%%%%%%%%%%%%%%%%%%%%%%%%%%%%%%%%%%%%%%%%%%%%%%%%%%%%%%%%%%%%%%%%%

The prevalence of machine consolidation through the use of virtual machines (VMs) necessitates improvements in VM performance. On the architecture side, major processor vendors have introduced virtualization extensions to their instruction set architectures~\cite{popek1974formal,intel,armv8}.
The emergence of high-speed (10Gbe and faster) networking interfaces also requires that VMs be granted direct access to  physical devices, thereby eliminating the costly, software-based hypervisor device multiplexing.

Enabling untrusted VMs to directly access physical devices, however, compromises system security.
To overcome the fundamental security issue, the PCIe standard was extended to support self-virtualizing devices through the Single-Root I/O Virtualization (SR-IOV) interface~\cite{pcisigiov}).
This method enables a physical device (\emph{physical function} in SR-IOV parlance) to dynamically create virtual instances (\emph{virtual functions}). Each virtual instance receives a separate address on the PCIe interconnect and can, therefore, be exclusively assigned to, and accessed by, a specific VM. This method thereby distinguishes between the physical device, managed by the hypervisor, and its virtual instances used by the VMs.
%
Importantly, it is up to the physical device to interleave and execute requests issued to the virtual devices.

Self-virtualizing devices thus delegate the task of multiplexing VM requests from the software hypervisor to the device itself.
The multiplexing policy, however, depends on the inherent semantics of the underlying device and must, naturally, isolate request streams sent by individual virtual devices (that represent client VMs).
For some devices, such as networking interfaces, the physical device can simply interleave the packets sent by its virtual instances (while protecting the shared link state~\cite{smolyar15sriovsec}).
%performance of direct assignment with flexibility of emulation
%
However, enforcing isolation is nontrivial when dealing with storage controllers/devices, which typically store a filesystem structure maintained by the hypervisor. The physical storage controller must therefore enforce the access permissions imposed by the filesystem it hosts.
%
%SHARON?
The introduction of next-generation, commercial PCIe SSDs that deliver multi-GB/s bandwidth~\cite{intel-ssd,seagate16ssd}) emphasizes the need for self-virtualizing storage technology.

In this paper we present NeSC, a self-virtualizing nested storage controller that enables hypervisors to expose files and persistent objects\footnotemark (or sets thereof) as virtual block devices that can be directly assigned to VMs. NeSC enforces the permissions set by the hypervisor and the hosted filesystem and prevents virtual devices (and VMs) from accessing stored data for which they have no access permissions.

\footnotetext{We use the terms files and objects interchangeably in this paper.}

NeSC enforces isolation by associating each virtual NeSC device with a table that maps offsets in the virtual device to blocks on the physical device. This process follows the way filesystems map file offsets to disk blocks.
%
VMs view virtual NeSC instances as regular PCIe storage controllers (block devices). Whenever the hypervisor wishes to grant a VM direct access to a file, it queries the filesystem for the file's logical-to-physical mapping and instantiates a virtual NeSC instance associated with the resulting mapping table.
%
Each access by a VM to its virtual NeSC instance is then transparently mapped by NeSC to a physical disk block using the mapping table associated with the virtual device (e.g., the first block on the virtual device maps to offset zero in the mapped file).

We evaluate the performance benefits of NeSC using a real working prototype implemented on a Xilinx VC707 FPGA development board. Our evaluation shows that our NeSC prototype, which provides 800MB/s read bandwidth and almost 1GB/s write bandwidth, delivers \speedup{2.5} and \speedup{3} better read and write bandwidth, respectively, compared to a paravirtualized \emph{virtio}~\cite{russell2008virtio} storage controller (the de facto standard for virtualizing storage in Linux hypervisors), and \speedup{4} and \speedup{6} better read and write bandwidth, respectively, compared to an emulated storage controller.
We further show that these performance benefits are limited only by the bandwidth provided by our academic protoype. We expect that NeSC will greatly benefit commercial PCIe SSDs capable of delivering multi-GB/s of bandwidth.

%%%%%%%%%%%%%%%%%%%%%%%%%%%%%%
% moved here from motivation section.
%%%%%%%%%%%%%%%%%%%%%%%%%%%%%%
\begin{figure*}[!ht]
  \centering
    \subfloat[Emulation ]{
      \includegraphics[width=0.3\textwidth]{figs/emulation.pdf}
      \label{fig:storage:emulation}
    }
    \hfill
    \subfloat[virtio]{
      \includegraphics[width=0.3\textwidth]{figs/virtio.pdf}
      \label{fig:storage:virtio}
    }
    \hfill
    \subfloat[Direct-IO]{
      \includegraphics[width=0.3\textwidth]{figs/direct.pdf}
      \label{fig:storage:direct}
    }
    \caption{IO Virtualization techniques.
      \label{fig:storage}}
    
\end{figure*}
%%%%%%%%%%%%%%%%%%%%%%%%%%%%%%


Although this paper focuses on the performance benefits that NeSC provides for VMs, it is important to note that NeSC also provides secure and direct storage access for accelerators connected to the PCIe interconnet (e.g., GPGPUs, FPGAs). As virtual NeSC instances are directly accessible on the PCIe interconnect, they can be accessed directly by other PCIe devices (using direct device DMA), thereby removing the CPU from the accelerator-storage communication path.

In this paper we make the following contributions:
\begin{itemize}
\item
  We introduce NeSC, a self-virtualizing, nested storage controller that provides VMs with direct and secure storage access without hypervisor mediation.

\item
  We propose a hardware implementation for NeSC, which is prototyped using a Xilinx VC707 (Virtex-7) FPGA development board.

\item
  We evaluate benefit of self-virtualizing storage using microbenchmarks and complete applications. Our evaluation shows that the NeSC prototype delivers \speedup{2.5} and \speedup{3} better read and write bandwidth, respectively, than state-of-the-art software virtual storage.
  
\end{itemize}


The rest of this paper is organized as follows:
Section~\ref{sec:motiv} discusses the performance overheads of storage virtualization, and  Section~\ref{sec:related} discusses related work.
We present the NeSC system design in Section~\ref{sec:design} and its architecture in Section~\ref{sec:arch}. We then present our evaluation methodology in Section~\ref{sec:methodology}, the evaluation in Section~\ref{sec:eval}, and conclude in Section~\ref{sec:conclusions}.



\hide{

Emerging high-performance data processing systems make use of multiple processors and accelerators
and compartmentalize applications inside virtual machines \cite{huang2015unified}.
These rich environments are thus composed of many virtual entities that need fast access to disk storage.
Future storage technologies such as NAND flash and phase-change memories will reduce the latency and increase the bandwidth moving the storage bottleneck to the software layers.

I/O virtualization solutions are emulation and direct assignment \cite{popek1974formal}..
When using emulation the isolation of the physical device is done by the hypervisor, using a file on its filesystem emulating the virtual machine's disk. Each time the virtual machine
tries to access its storage, the CPU traps into the hypervisor witch will route the request to an offset in the file. The mapping from file offset to storage block is done by the hypervisor.
In this methods there is a performance overhead because every request must be taken care of by the hypervisor.
Another option is using direct assignment, in this method the devices PCIe address space is mapped to a single virtual machine allowing it to access the device with near native performance.
SRIOV protocol allows devices such as VNMe devices to present themselves as multiple virtual devices on the PCIe address space letting the hypervisor to assign each virtual device to a virtual machine.
In this method isolation is guaranteed at the partition level by the device, each virtual machines request to the device is done through its attached virtual function and multiplexing of the requests is done
by the device. Isolation at the partition level has better performance but is less flexible than using a file.

In this paper we present vdisk, a storage controller that implements PCIe SRIOV protocol and gives the hypervisor the flexibility of an EXT4 filesystem over the device. 
}


%%%%%%%%%%%%%%%%%%%%%%%%%%%%%%%%%%%%%%%%%%%%%%%%%%%%%%%%%%%%%%%%%%%%%%%%
\section{On the performance overheads of storage virtualization}
\label{sec:motiv}
%%%%%%%%%%%%%%%%%%%%%%%%%%%%%%%%%%%%%%%%%%%%%%%%%%%%%%%%%%%%%%%%%%%%%%%%

Hypervisors virtualize local storage resources by mapping guest storage devices onto files in their local filesystem, in a method commonly referred to as a nested filesystem~\cite{le12nested}. As a result, they replicate the guest operating system's (OS) software layers that abstract and secure the storage devices. Notably, these software layers have been shown to present a performance bottleneck even when not replicated~\cite{yu14swoverheads}, due to the rapid increase in storage device bandwidth ~\cite{intel-ssd,seagate16ssd}.
%
Moreover, further performance degradation is caused by the method by which hypervisors virtualize storage devices and the resulting communication overheads between the guest OS and the underlying hypervisor.
Consequently, the storage system is becoming a major bottleneck in modern virtualized environments.
%
In this section we examine the sources of these overheads and outline how they can be mediated using a self-virtualizing storage device.

% software layers
Prevalent storage devices present the software stack with a raw array of logical block addresses (LBA), and it is up to the OS to provide a flexible method to partition the storage resources into logical objects, or files. In addition, the OS must enforce security policies to prevent applications from operating on data they are not allowed to operate on.
%
The main software layer that provides these functionalities is the filesystem, which combines both allocation and mapping strategies to construct logical objects and map them to physical blocks (for brevity, we focus the discussion on these two functionalities and ignore the plethora of other goals set by different filesystems). In addition, the filesystem layer also maintains protection and access permissions. Besides the filesystem, another common layer is the block layer, which caches disk blocks and abstracts the subtleties of different storage devices from the filesystem layer.

When an application accesses a file, the OS uses the filesystem layer to check the access permissions and map the file offset to an LBA on the storage device. It then accesses its block layer, which retrieves the block either from its caches or from the physical device. In a VM, this process is replicated since the storage device viewed by the guest OS is actually a virtual device that is mapped by the hypervisor to a file on the host's storage system. Consequently, the hypervisor invokes its own filesystem and block layers to retrieve the data to the guest OS. 

% virtualization method
Figure~\ref{fig:storage} illustrates the three most common methods by which hypervisors virtualize storage devices:

\begin{enumerate}
\item
  \emph{Full device emulation}~\cite{sugerman2001virtualizing} (Figure~\ref{fig:storage:emulation})\quad 
  In this method, the host emulates a known device that the guest already has a driver for. The host traps device accesses by the VM and converts them to operations on real hardware. The emulated device is represented as a file on the host filesystem, and whenever the guest tries to access a virtual LBA on the device, the host converts the virtual LBA to an offset in the file.

\item
  \emph{Paravirtualization}~\cite{barham2003xen,russell2008virtio} (Figure~\ref{fig:storage:virtio})\quad
  This method, commonly referred to as virtio after its Linux implementation, eliminates the need for the hypervisor to emulate a complete physical device and enables the guest VM to directly request a virtual LBA from the hypervisor, thereby improving performance. This is the most common storage virtualization method used in modern hypervisors.

\item
  \emph{Direct device assignment}~\cite{raj2007high} (Figure~\ref{fig:storage:direct})\quad
  This method allows the guest VM to directly interact with the physical device without hypervisor mediation. Consequently, it delivers the best storage performance to the VM. However, since storage devices do not enforce isolation among clients, it does not allow multiple VMs to share a physical device.
\end{enumerate}

%%%%%%%%%%%%%%%%%%%%%%%%%%%%%%
\begin{figure}[t]
  \centering
  \includegraphics[width=\columnwidth]{figs/motivation.pdf}
  \caption{The performance benefit of direct device assignment over virtio for high-speed storage devices. Fast devices were emulated using an in-memory disk (ramdisk) whose bandwidth, due to the overheads of the software layers, peaks at 3.6GB/s.
    \label{fig:directperf}}
  \end{figure}
%%%%%%%%%%%%%%%%%%%%%%%%%%%%%%

% quantify the overheads
Figure~\ref{fig:directperf} quantifies the potential bandwidth speedup of direct device assignment over the common virtio interface for high-speed storage devices. We have emulated such devices by throttling the bandwidth of an in-memory storage device (ramdisk). Notably, due to OS overhead incurred by its software layers, the ramdisk bandwidth peaks at 3.6GB/s.

The figure shows the raw write speedups obtained using direct device assignment over virtio for different device speeds, as observed by a guest VM application.
Notably, we see that compared to the state-of-the-art virtio method, direct device assignment roughly doubles the storage bandwidth provided to virtual machines for modern, multi GB/s storage devices.
The reason for these speedups is that as device bandwidth increases, the  software overheads associated with virtualizing a storage device become a performance bottleneck.
Using the direct device assignment method eliminates both the virtualization overheads as well as the overheads incurred by the replication of the software layers in the hypervisor and the guest OS.

% we need NeSC
The potential performance benefits of direct device assignment motivate the incorporation of protection and isolation facilities into the storage device. These facilities will enable multiple guest VMs to share a directly accessed physical device without compromising data protection.

%%%%%%%%%%%%%%%%%%%%%%%%%%%%%%
\begin{figure}[t]
  \centering
  \includegraphics[width=\columnwidth]{figs/nesc-overview.pdf}
  \caption{Exporting files as virtual devices using NeSC.\label{fig:nesc_outline}}  
\end{figure}
%%%%%%%%%%%%%%%%%%%%%%%%%%%%%%

This paper presents the nested storage controller (NeSC), which enables multiple VMs to concurrently access files on the host's filesystem without compromising storage security (when NeSC manages a single disk, it can be viewed simply as a PCIe SSD).
Figure~\ref{fig:nesc_outline} illustrates how NeSC provides VMs with secure access to a shared physical device.
NeSC leverages the SR-IOV features of the PCIe gen3~\cite{pcisigiov} to export host files as multiple virtual devices on the PCIe address space. Each virtual device is associated with a collection of non-contiguous blocks on the physical device, which represent a file, and is \comment{sharon v3. I think we are ok here} prevented from accessing other blocks. VMs can therefore directly access the virtual device and bypass the virtualization protocol and hypervisor software (notably, NeSC is compatible with the modern NVMe standard~\cite{nvme}).
The rest of this paper describes the design, architecture, and evaluation of NeSC.




%% Storage devices inherently maintain state and do not enforce isolation, requiring software to do so.
%% - client requests the OS to fetch a file/object. OS uses filesystem to logically partition storage to multiple files/objects
%% - Storage devices were traditionally slow, so software overhead was not an issue
%% - Prevalent PCIe and NVMe storage devices are much faster, so having the software on the critical path has become an issue
%% - We need direct device access to ameliorate the overheads of virtualization, but that requires that devices transparently enforce isolation between clients
%% - need to maintain the flexibility provided by the filesystem abstraction%


%% \cite{nested-filesystems}


%% % discuss this vs. partition-based direct accesses
%% filesystems provide management flexibility


%% 1. VMs need direct access to files (image file, data files)
%% - esp. in the context of emerging memory technologies \cite{huang2015unified}.
%% - files/object are a dynamically flexible storage abstraction
%% - easy to change content, size
%% - easy to move between machines (as opposed to partitions)

%% 2. hardware has no notion of files/objects, so it provides no isolation guarantees
%% - VM can access an entire disk/partition, not part thereof.
%% - object-level isolation must be maintained at the software level, with substantial overheads
%% - *Figure* depicted storage virtualization using existing designs (emulation, paravirt, direct device I/O)

%% 3. we introduce hardware-based object virtualization
%% - hardware enforces isolation at wire speed (gets software off the critical path)
%% - maintains flexibility and generality of traditional filesystems.


%% - SR-IOV alone does not provide what we need
%%   - SR-IOV allow to dynamically create logical devices.
%%   - it does not say anything about what is the meaning of the ``logical device'' (e.g., the relation between a physical NIC and a logical NIC is different than the relation between a physical storage device and a logical one)

%% Each vdisk is a represented as a file so easy to migrate  
%% asd


%% discuss the potential performance benefits due to nested filesystems \cite{le12nested}

%% %%%%%%%%%%%%%%%%%%%%%%%%%%%%%%%%%%%%%%%%%%%%%%%%%%
%% \subsection{Supporting direct accelerators-storage accesses}
%% %%%%%%%%%%%%%%%%%%%%%%%%%%%%%%%%%%%%%%%%%%%%%%%%%%
%% save power and not need to interfere with CPU 

%%%%%%%%%%%%%%%%%%%%%%%%%%%%%%%%%%%%%%%%%%%%%%%%%%
\section{Related Work}
\label{sec:related}
%%%%%%%%%%%%%%%%%%%%%%%%%%%%%%%%%%%%%%%%%%%%%%%%%%

NeSC was motivated by the necessity to redesign the hardware/software storage stack for modern, high-speed, self-virtualizing storage devices. In this section we examine related research efforts.

The Moneta~\cite{caulfield10moneta} and Moneta-D~\cite{caulfield12moneta-d} projects both try to optimize the I/O stack by introducing new hardware/software interfaces.
The Moneta project introduces a storage array architecture for high-speed non-volatile memories. Moneta-D extends the design to allow applications
to directly access storage by introducing per-block capabilities~\cite{levy1984capability}. When an application directly accesses the storage, permission checks are done by the hardware to preserve protection policies dictated by the OS.
The design, however, requires applications to use a user space library to map file offsets to the physical blocks on the device.
In contrast, NeSC enforces file protection by delegating file mapping to the device level rather than to the application level.

Willow~\cite{seshadri2014willow} examines offloading computation to small storage processing units (SPU) that reside on the SSD. The SPUs run a small OS (SPU-OS) that enforces data protection. Willow implements a custom software protocol that maintains file mapping coherence between the main OS and the SPU-OS instances.

NVMe~\cite{nvme} is a new protocol for accessing high-speed storage devices. Typically implemented over PCIe, NVMe defines an abstract concept of \emph{address spaces} through which applications and VMs can access subsets of the target storage device. The protocol, however, does not specify how address spaces are defined, how they are maintained, and what they represent. 
NeSC therefore complements the abstract NVMe address spaces and enables the protocol to support protected, self-virtualizing storage devices.


MultiLanes~\cite{kang2014multilanes}
mediates the contention on a hypervisor's shared I/O stack by presenting each VM with a dedicated, virtualized I/O stack. Nevertheless, the underlying device accesses issued by each VM must still be mediated, in software, by the hypervisor.


Peter et al.~\cite{peter2016arrakis} 
  argue that the SR-IOV concept should be extended and that the operating system should serve as a system's control plain and enable applications to directly access devices. The NeSC storage controller is in line with their view.


Finally, FlashMap~\cite{huang2015unified} provides a unified translation layer for main memory and SSDs, which enables applications to map files to their memory address space. Specifically, FlashMap unifies three orthogonal translation layers used in such scenarios: the page tables, the OS file mapping, and the flash translation layer.
FlashMap, however, only supports memory-mapped SSD content, and it does not address how the unified translation would support virtualized environments.

In summary, NeSC's self-virtualizing design is unique in that it delegates all protection checks and filesystem mappings to the hardware. Furthermore, its  filesystem-agnostic design preserves the hypervisor's flexibility to select and manage the host filesystem.




\input{vdisk-design}
\input{vdisk-arch}
\input{vdisk-methodology}
\input{vdisk-eval}
\input{vdisk-conclusions}
%###############################################################################


%###############################################################################
\hide{
  \section*{Acknowledgments}
  \label{sec:acks}
}

%###############################################################################

%\small
\bibliographystyle{ieeetr}
\bibliography{macros,vdisk}

\end{document}

